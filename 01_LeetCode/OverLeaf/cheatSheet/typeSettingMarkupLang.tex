

⭐ Keynotes for Beginners Learning LaTeX (.tex Markup Language)]

1. LaTeX is not a word processor — it’s a typesetting language

    You write code, and LaTeX renders it into a clean, professional PDF.
    Ideal for resumes, research papers, theses, reports, CVs.




2. Everything is written in plain text
    A .tex file is simply a text file with commands.
    Example:
    \textbf{Bold Text}



3. Structure is created with commands
    Documents are built from sections:
    \section{Introduction}
    \subsection{Background}



4. Packages extend functionality
        Use \usepackage{} to add new features:
            geometry → change margins
            hyperref → clickable links
            enumitem → custom bullets
            fontspec → custom fonts


5. LaTeX compilers matter
    pdfLaTeX → default, limited fonts
    XeLaTeX → use system fonts (Times New Roman, etc.)
    LuaLaTeX → advanced typesetting
                 Resume templates usually need XeLaTeX.




6. Everything is controlled with environments
    Examples:
    \begin{itemize}
    \item Point 1
    \item Point 2
    \end{itemize}

Produces bullet points.


7. Precise control over formatting
Instead of manual formatting like MS Word:
    LaTeX defines spacing, alignment, indentation, and typography programmatically.
    Ensures consistency across entire document.


8. Templates save time
You don’t start from zero — reusable templates exist for:
    Resumes
    Research papers
    Journals (IEEE, ACM)
    Assignments
    Thesis/Dissertations
A beginner only needs to edit text, not design layouts.



9. LaTeX avoids formatting errors
    No accidental:
        Shifting margins
        Broken headings
        Wrong numbering
        Misaligned bullets

    LaTeX handles all formatting automatically.


    